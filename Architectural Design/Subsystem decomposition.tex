
\paragraph{}
In order to build the eCoach program, this task has to be divided in smaller sub problems. So the system is divided into subsystems. To increase the cohesion of the different subsystems, a strong division is needed. Another important issue is reuse, will parts of the program be reusable. To ensure the quality of the program the system also has to be easy to test. To achieve these goals, the choice has been made to design he system by use of a \gls{model-view-controller}-model (MVC). Cohesion is achieved  with the strong layer-cohesion. The model-layer will be easy to reuse, as well as parts of the control-layer. And the system will be easy to test because the application can be tested separately from the user interface. One more reason for this approach is the need of two different representations of the data for the therapist and the patient: two different \textit{views} on the same data.
\paragraph{}
So for the view-layer at least two different views will be created, one for the therapist and one for the patient. The patient view-layer will  use an avatar as an eCoach and in this project the avatars display is realized by remote procedure calls to an engine called vizard. To ease the replacement of this engine, communication from the view subsystem with this system has to be done through an interface. This interface is a sub system of the view layer.
\paragraph{}
The controller-layer can be divided into the different tasks the system has to accomplish. A communication system to transfer data from the patient to the therapist and back is one of the subsystems. Another subsystem is the behaviour logic of the coach avatar. The eCoach has to behave as a real coach would do. 
\paragraph{}
The model-layer consists of the database system to store all the information. For now, the model-layer is not sub divided. In section~\ref{sec:datamap} a further description of this layer is given.
