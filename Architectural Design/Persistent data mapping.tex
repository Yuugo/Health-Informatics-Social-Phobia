\section{Persistent data mapping}
\label{sec:datamap}
\paragraph{}
For this project we will have to use a database of some sort, because there is a lot of data that has to stored and accessed later on. There are many ways to store data for example a file system, but the team has chosen for a database because of the easy built-in functions offered by a database, such as rollbacks and faster searching with the use of indexes.
This data may also have to be accessed at an other location for example, the patients progress has to be viewed by the therapist and the therapist needn't be at the same location. The team's preference for a database type is a SQL database that is "in the cloud" for when we deploy the system. When we're developing the system we'll use local databases on our laptops. 
We have chosen for this type because we all have experience with SQL. We have chosen for "in the cloud" because then it will be accessible everywhere and we have no maintenance for the server and we don't even have to buy a server or domain etc. 
In the database all the therapists and patients are stored, each therapist has a reference to which patient he/she has. Also all login information is stored for everyone (login names and passwords). This is done, so that the therapists and patients can login everywhere, for example to see an overview or their progress. The progress is very important for the therapist, because all exposure session are now digital so the patient and therapist don't see each other on a regular basis.
The social events that the patient talks about with his/her eCoach will also be saved , so the avatar can 'remember' it and talk about it with the patient some time in the future.
\paragraph{}
Because the server or the patients internet connection can be offline, the patients progress and everything else that can be stored in the database, is first saved locally at the patients hard drive. And then it is synced with the server.