De werkhouding binnen de groep was heel erg 'to the point', vergaderingen waren altijd erg kort en bondig. De voorkeur lag bij het vlug verdelen van de taken en dan aan de slag te gaan. Over de punten waar een beslissing over gemaakt moest worden werd vaak snel een compromis gesloten. Dit zorgde er in ieder geval voor dat de sfeer nooit verpest werd, maar het zorgde soms wel voor een wat afstandelijkheid, iedereen doet toch zijn eigen ding. Het eerste kwartaal, waarbij vooral veel verslagen gemaakt moesten worden en een beeld geschept worden van wat het programma moet gaan inhouden, hebben we elke bijeenkomst wel een vergadering gehad of dingen besproken. Dit werd minder toen het programmeren begon in het tweede kwartaal. De vergaderingen hadden niet veel structuur. Er werd gevraagd wat er nog moet gebeuren, werden wat dingen opgeschreven. Soms zorgde dit er wel voor dat de planning niet erg duidelijk was, maar het uiteindelijke resultaat met de groep was altijd op tijd en vaak ook goede kwaliteit. 
Met projectvaardigheden hebben we een testje gedaan om te kijken wat voor persoon je bent in een team. Hierbij hadden we allemaal 'bedrijfsman' erg hoog staan. Dit is iemand die praktisch denkt, beslissingen in concrete werkzaamheden omzet en gedisciplineerd kan werken onder druk. Maar dit is ook iemand die te snel aan de slag kan willen; beginnen met ordenen en regelen terwijl de doelen en uitkomsten nog niet precies vastgesteld zijn. Deze rol lijkt erg goed te kloppen met hoe de groepsdynamiek was. Snel aan de slag, maar wel beheerst en gedisciplineerd. Misschien was het handig geweest als er aan het begin van het project, toen het duidelijk werd welke rollen we hadden, iemand bijvoorbeeld hadden aangesteld als voorzitter, zodat er wat meer diversiteit zou zijn in de groep. 
Uiteindelijk is het project prima verlopen en hebben we een prettige samenwerking gehad.

The posture within our group was very much 'to the point'. The meetings were often brief. The preference was to make 