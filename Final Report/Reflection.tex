The posture within our group was very much 'to the point'. The meetings were often brief. The preference seemed to be to quickly divide the tasks and then to go to work. For the points that needed discussion there was often quickly a compromise. There were never any real conflicts actually. This made that the atmosphere was generally very positive, but sometimes a little detached because everyone was doing his own work.
The first quarter, where the main focus was on reports and understanding what our program had to do, we had regular meetings practically every project day. This declined during the second quarter, when the actual programming began. The meetings lacked somewhat in structure. We quickly discussed what had to be done and someone wrote some points down, without an agenda. This seemed to work pretty well for us actually. It sometimes made that the planning was not quite clear, although we were always comfortably on time with deadlines.
At the beginning of the first quarter we had a weekly course to learn about working in a group. In this course we had to do a little test to find out what kind of team player you are. The result of our test were more or less the same, with 'implementer' high on all our test results. An 'implementer' is someone who thinks practically, turns decisions into actions and is efficient and self-disciplined. But it is also someone who wants to start working too soon; to begin planning and working before a concrete plan is made. It seems that this role is reflected in our group dynamic. No long meetings, quickly to work but the work is done good. In hindsight, it could have been a good idea to have appointed someone as co-ordinator for example, so we would have a more diverse group.
In the end the project went well, and everyone had a positive collaboration.