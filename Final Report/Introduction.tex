Every development project needs guidelines and preparation to be successful. These ideas need to be formed before the project itself starts off, which means that the team is required to have an idea of what the development of the product will be like. It might be the case that the team doesn't actually know for certain what is expected of them, so the will have to make an effort to estimate the process to the best of their abilities.

The goal of this document is to lay out the results of our development process. In the past months we have made a large amount of decisions to try and keep our work organized and stable. Now that we are almost done with the process and have to required program working, we will look back on those decisions and evaluate whether they were the right choice to make and learn from possible mistakes we made.

This will be done by first covering major issues we hit on the way to the end. When a team that isn't quite as experienced in the field starts a project like this, they will run into problems that they don't know the solution to. The first chapter will cover the solutions we have found to this kind of problem. Next, we will discuss how the team worked as a whole. A team project is mostly dependant on how the members of the team are capable of working together towards the goal of their project. Reflecting on the team synergy will offer learning points for future projects.
After that, each team member will pose their individual reflections on the project. They will cover what their overall opinion on the assignment was, what they have added to the project and what they have learned from this.
Finally, a collection of SCRUM plans is included to show our development process and how we decided to plan things. This gives an overview of what parts of the final implementation took a long time to create and in what order the functionality was implemented.
