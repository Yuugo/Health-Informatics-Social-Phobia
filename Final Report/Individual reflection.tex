\section{Reflection Koos van der Linden}
In general, this project was a little disappointing. The main reason for this is that my presumptions and expectations of the project were quite wrong. One of my expectations for example was that we should have to work with virtual reality and 3D modelling. My expectations turned out to be unrealistic during the project, but not until later in the project because the first part of the project was all about documenting. 
No virtual reality, no 3D, but a lot of trouble with getting the avatar right in the gui. Most of my time in this project has been spent on this problem. And when I solved this problem on my computer I have been busy for another while to get it working on other computers. I also had to refactor the code to make it more usable. Eventually all this code is based very specific on the program of the avatar and because of that not portable, but if the same avatar is used, I have tried to increase portability and usability as much as possible.
I also have been busy with creating a sort of gui testing 'framework'. This was in the beginning of the project and I had no C\# experience yet. By creating this piece of code I have learned a lot of C\#, which I enjoyed, because learning new programming languages and its possibilities broadens your view. One of the most interesting things to learn about C\# was reflection, the possibility to inspect and alter already compiled and running code. I am sure this will be useful later for something.
I have succeeded to solve these problem and this also one of the reasons that I am content about my own contribution to the project. I have been able to finish my tasks on time with reasonable quality. The peer review also learned me that others were content with my contribution.
The team working was good in general. In the beginning I thought that we lacked commitment and structure, but later in the project this improved. We still didn't have a good structure but the group atmosphere was good. With lack of structure I for example, mean that we didn't have fixed moments for meetings. Also the planning was not good. We didn't use planbox in the beginning of the project for example.
During this project I have learned a lot, but other things than I expected. This project was longer than the previous on the TU Delft, and made this project one multiple ways different from those previous projects. I also learned a few basic principles of scrum, which seems useful for longer projects and teamwork. 

\section{Reflection Elgar de Groot}
One of the first things I realised about this project is that it's not just about writing quality software, it is about learning to work in a group on some software. There was a lot of focus on the lifetime of a software project. When I look at just the software we had to write, I am not really exited about it. A lot of times when I thought I knew what the purpose of the program was, I was proven to be wrong. This was a little frustrating for me at first, because I had now idea what we were meant to write. I got over it and accepted it for what it was, and that seemed to work. It went better with time and had more and more fun with it. Overall, from this perspective, it was ok.
But when I look at it as a chance to learn to work in a group on a piece of software that someone else requests with vague language, it is a great learning chance. There is a fat chance that this kind of situation will occur when working in the industry. You have to write some software that you don't really find interesting. You have to work in a group of random people and have to make it work. You don't have a detailed description of what the program has to do. So from this perspective I think it is a great asset to have done this project.

\section{Reflection Johnny Verhoeff}
When the project began, we first had a lot of seminars about all the different contexts. And I said to myself that I wanted to do the context with the avatar because I thought that was the coolest. While it was the coolest, it wasn't that easy to embed the avatar in de C\# GUI. I started to implement it but couldn't do it alone so another team member helped me with that. In this project we had to work with scrum, which for me was the first time I had to do that. I found that this planning worked better then traditional planning, we always had a working system which was nice. 
We had to write a couple of documents at the beginning of the project before we even had written some code. The documents had to be updated during the project so we always had an up-to-date version of all documents. But one problem with this was that we didn't really know what we were going to build and what our limitations were. So I wrote about all sorts of cool stuff that we could do such as a local copy at the patient's computer for when they are not connected to the Internet. That would automatically sync with the main database. This was proven to be too hard to implement and we didn't have the time do to it, so we scratched that idea. 
This project was the first time that I programmed in C\# and the first time I worked with Visual Studio.  In the past I had coded in Java and I found that C\# was very similar to Java, so that wasn't a major issue. Visual Studio was a nice IDE to develop in, but it's a shame that the free version is very limited, so halfway into the project we had to switch to the evaluation version of Visual Studio Ultimate. We needed to do this because else we couldn't connect to the database.
We could not find a good GUI testing tool for the free version of Visual Studio so one of the team members created one with bits he found on StackOverflow. I used this GUI test to write a GUI test for one of the forms.  One downside was that it uses reflection so it depended heavily on the names of the objects in the form. If one of the names in the form would change then the test failed and had to be manually fixed.

All things considered, the working environment was great. The teamwork was great, often we would joke around and we could all get along.
I would gladly do another project with this team.

\section{Reflection Hugo Reinbergen}
When this project just started I had high expectations of what we were supposed to build. Using virtual reality to support the treatment of patient with social anxiety sounds like a great field to develop software for. Sadly it turned out there wasn't much virtual reality in place. The project was actually about creating an interface that displayed an avatar that was already created for us. So actually it was a project about creating a GUI and handling an external process in that GUI. This isn't exactly a bad thing, but it's not what I expected to be making.

Anyway, in the end I'm looking back at the project with content. Though it wasn't exactly what I expected, I can say that I have learned a lot from it. During this project I've focused mainly on the back-end of the program. More concretely, I've created a XML to Object parser, set up a database and created the functionality to access that database smoothly from within the program. With this, I've mainly improved my overall skills in programming in C\# and understanding relational database mapping. 

The project was set up to be Test Driven Scrum. This meant we had to work in sprints of 2 weeks where we made small improvements every iteration. This helped me understanding the value of proper planning and guessing the time required to implement certain features. This will be of great help in future projects since distributing your available time increases your efficiency which in turn makes you more valuable for your cost.

\section{Reflection Willem Vaandrager}

