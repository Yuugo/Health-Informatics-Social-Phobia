\chapter{Introduction}

\paragraph{}
The problem of Social Phobia in our society has been prevalent for a considerable time. It ranges from people feeling uneasy when speaking in front of a crowd to individuals being unable to leave their houses because of their anxiety. 
A problem with curing this type of phobia is that interaction with patients can be difficult. If someone is afraid to talk to people, it is possible that to make the step to talk to a therapist for help is very large. A possible solution to this problem is an \gls{eCoach}, device or program that can substitute for a therapist. Using an \gls{eCoach} can create an easier environment for the patient to be treated in and can make the decision to ask for treatment easier and can speed up the process.
\paragraph{}
The goal of this report is to describe the vision on how the process of creating an \gls{eCoach} works. The main question asked are: "What does the user want to do with the product?", "How does the communication between the therapist and patient work" and "What is the global planning for delivering working versions of the program?".
\paragraph{}
The report will start of with laying out the product. This will consist of the Product Vision, which describes who the customers are and what they need, as well as a high-level product backlog that contains epics to describe the vision and closing off with a Roadmap with the planning of major releases of the product and what the goals are for those releases.
Next is the actual Product Backlog. This will contain a variety of user stories about features, defects, technical improvements and know-how acquisition. It will also cover the initial release plan.
Then the definition of when this product is considered done is discussed. What requirements have to be met to consider the product finished and what are the milestones along the way to reaching that goal.
The report will close off with a Glossary that contains the definitions of technical terms used throughout the document.