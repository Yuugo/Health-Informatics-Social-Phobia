\chapter{Product}
This chapter will provide a basic overview of what the product will become. The first section describes the team's vision on the product's behavior and what purpose it serves. In the next chapter this vision is refactored in some high-level product backlog items whom can then be further refined into the product backlog. These high-level product backlog items are then prioritized into a roadmap which serves as an indicator of when the team wants the items to be implemented.

\section{Product vision}
For Social phobia therapist who want to provide patients with a home treatment system, \gls{eCoach} is a  solid solution for aiding patients with minimal interaction with the therapist. To ease treatment of social phobic patients, virtual exposure is an ideal solution, for the following reasons:
\begin{itemize}
\item It improves the comfort of the patients, because there is no need to leave their home regularly.
\item It reduces costs by minimalizing interaction with the therapist
\item It simulates hard to realize situations in in vivo therapy such as presenting for a crowd or waiting at a bus stop.
\end{itemize}
Currently only virtual therapy systems exist that require a therapist to be present and make the \gls{avatar} interact with the patient by listening to the patient and selecting the appropriate response. These systems are not sophisticated enough to be used at home and in large scale because the presence of a therapist is still required. To counter this problem attempts have been made to develop an virtual coach to guide the patients with their therapy at home.
\gls{eCoach} will help, stimulate and inform the patient, but also give feedback and allows the patient to communicate the results and progress with the therapist. The results of the exposures are automatically shared with the therapist. This will result in improved adherence of the patient because he/she knows that the therapists monitors the progression.
Furthermore, the \gls{eCoach} gives the patient advice to define sub-goals that are easier to realize than the main goal. Before an exposure session, the \gls{eCoach} \gls{avatar} will clearly instruct the patient about the session to improve the patients efficacy. After an exposure session, the \gls{eCoach} will reflect on the session with the patient and will provide the patient with an overview of his anxiety levels during the session. When appropriate, suggestions for improvement will be given.
Because the \gls{eCoach} system has to be used by people with different levels of computer skills, the system will have to be user friendly and adaptable to the level of experience of the user. The \gls{eCoach} \gls{avatar} will help the user to use the system by giving tips and suggestions at the appropriate moments.


\section{High-level product backlog}
\paragraph{}
Anxiety questionnaire: \\
Form, which allows the system to gain information about the patient's current anxiety status. As an online form or the \gls{avatar} will ask the questions, which is more personal and could be more helpful.
\paragraph{}
Communication server: \\
This allows the patient and the therapist to communicate online and will also send and retrieve the patient's files, progress and therapist's results.
\paragraph{}
Discussing events outside therapy: \\
The \gls{avatar} will talk with the patient about social contact the patient has had outside the therapy.
The system will gain more information about the patient's usual behaviour and might open up the patient some more if the patient can just talk with the \gls{avatar}.
\paragraph{}
Overview of patient's progress in therapy: \\
This will show the progress of the patient in graphs, tables and animation, which will give a clear overview of the patient's progress. 
\paragraph{}
Suggestions and tips from the \gls{avatar}: \\
Tips and suggestions the \gls{avatar} will give to the patient based on the results and behaviour of the patient. 
The \gls{avatar} also discusses things the patient has to do for the next session according to the result of the previous session(s). It will give more specific tips and suggestions that will help the patient in his current situation.
\paragraph{}
Personalized \gls{avatar}: \\
The \gls{avatar} is able to give feedback as it suits the patient best. It has learned during the progress what is the most effective on the patient and will for instance adept to give more negative or more positive feedback based on those experiences.
\paragraph{}
Reflection of patient's progress in therapy: \\
The \gls{avatar} is able to interpret the results of the patient and will discuss the results. The \gls{avatar} discusses what the results mean and what the patient could do to improve or what the patient has done really well.

\section{Roadmap}
\begin{itemize}
\item Friday March 29: Anxiety questionnaire	
\item Friday April 5: Overview of patient's progress in therapy
\item Friday April 26: Communication server
\item Friday May 3: Reflection of patient's progress in therapy
\item Friday May 10: Personalized \gls{avatar}
\item Friday May 17: Suggestions and tips from the \gls{avatar}	
\item Friday May 24: Discussing events outside therapy
\end{itemize}
