\chapter{Persona}
\paragraph{Persona 4}
\begin{wrapfigure}[12]{r}{0.35\textwidth}
\centering
\includegraphics[width=0.3\textwidth]{"jan"}
\caption{Photo of Jan}
\end{wrapfigure}
\begin{itemize}
\item Name: Jan
\item Nationality: Dutch
\item Male, 42 years old
\item Super computer skills (coder)
\item Hardware: Laptop, pc, HDTV, tablet, smartphone
\item Does not like slow programs and difficult GUIs
\item Motivated to do "Home" treatment
\item Average self-effiacy
\item LAT (Living Apart Together) relation
\item Status: freelancer for his own company
\item Goal: Be able to give a presentation to his customers
\end{itemize}
\paragraph{Progress Reflection and Briefing Exposure Module (Block 6b+7): }
Jan starts his actual therapy session (according to his treatment schedule). He sets up all the necessary things to monitor his parameter: inserts the USB stick, put on the heart rate monitor device, setup the Bluetooth connection. Then, while he is starting eCoach, he complains why the setup is so complicated. After he logs in to eCoach, using a username and password, the avatar greets him and asks him to fill in a social anxiety questionnaire that his therapist has created earlier and send to the individual patients. \newline
After Jan finishes the social anxiety questionnaire, the program informs him that the therapist has send him a message. Jan clicks on the message box and reads the message. His therapist has send him a message about his progress that the therapist can monitor from the office. Jan says to him self, that he really likes that the therapist can monitor him without the therapist actually being in the same room. The therapist also says in the message that Jan can try out different reactions to a avatars response. Jan is complaining that he already tried this and that the therapist didn't need to remind him about his. \newline
Now he wants to se an overview of his progress to boost his own confidence. So he clicks on the button to show an overview of social anxiety scores he has entered over weeks. After viewing various scores of his fear levels and anxiety scores in exposure sessions the program asks him via an avatar  whether he is satisfied with his progress. Jan is still a bit worried about his progress and what the therapist thinks of his progress. Than the avatar steps in and explains that this can vary for each individual, the avatar comforts Jan and gives him positive feedback. After this Jan feels a lot better about himself and is now more determined to finish the therapy.\newline
The avatar invites Jan to indicate his expectations for next week. After reflecting on his progress, Jan likes to start with the actual exposure session. He clicks on this treatment plan. The avatar explains that today he will have to give a presentation in front of a small audience. The avatar also explains the goals of the exposure. Jan should look at the audience when he gives his presentation, and he should not try to hold something as a trick to keeps his hands from tremble. Next, the avatar ask Jan if he also has a goal in mind for this session and gives him a long list of possible goals. Jan select speaking without stuttering option. As stuttering is not under Jan direct control, the avatar suggests to change his goals into something he can actually control, for example, if he stutters to make a joke about it. Jan agrees that this is a goal he could try to achieve. After this the avatar screen disappears, the narrative text of the session appears on the screen and the exposure sessions starts. \newline
\newline
Claims:
\begin{enumerate}
\item Getting feedback from both the therapist and from the avatar about the previous exposure(s) will increase the chance of a patient to continue the therapy.
\item Patient’s awareness that therapist will regularly monitor their treatment progress improves patient adherence. 
\item Feedback from therapist about “events outside” the therapy will increase chance of patient to continue the therapy session.
\item Providing clear instructions about a session’s goal before exposure will improve efficacy of exposure.
\item Having an overview with interpretation of the progress so far will motivate patient to continue with the treatment.
\end{enumerate}
\paragraph{Current Exposure Reflection Module (Block 9):}

After the exposure session has ended, the avatar leads Jan into the reflection session. Here Jan is provided with an overview of his SUD scores and heart rate data. Also the times are shown for which Jan was talking and the avatar was talking. Jan thinks that this is a good representation of a session overview, he really finds this useful. \newline
The avatar also asks Jan to indicate whether he has applied any tricks (safety behavior). Jan is also asked to rate how much he achieved the goals for this sessions. \newline
After this the avatar suggest that Jan takes a rest for about five minutes, before he continues with the next exposure session, that the therapist has ordered. Then his screen turns black, only showing a clock counting backwards. When the five minutes are up, Jan sees the avatar once again and he is reminded about the goals for the next session. After this, Jan sees the narrative text again and a new exposure session starts. \newline

Claims: 
\begin{enumerate}
\item Having patient reflect on their exposure experience in between or after the last exposure session will improve the efficacy of the exposure.
\end{enumerate}


